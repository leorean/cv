\documentclass[9pt]{extarticle}

% \documentclass[10pt]{article} %Sets the default text size to 11pt and class to article.
%------------------------Dimensions--------------------------------------------
\pagenumbering{gobble}% Remove page numbers (and reset to 1)
\topmargin=-20pt %length of margin at the top of the page (1 inch added by default)
\oddsidemargin=0.0in %length of margin on sides for odd pages
\evensidemargin=0in %length of margin on sides for even pages
\textwidth=6.7in %How wide you want your text to be
\marginparwidth=0.5in
\headheight=0pt %1in margins at top and bottom (1 inch is added to this value by default)
\headsep=0pt %Increase to increase white space in between headers and the top of the page
\textheight=9.5in %How tall the text body is allowed to be on each page
 
\usepackage{xcolor}

\definecolor{linkcolor}{RGB}{105,0,0}

\usepackage[utf8]{inputenc}
\usepackage[colorlinks=true,urlcolor=linkcolor]{hyperref}

\newcommand\tab[1][1cm]{\hspace*{#1}}
\newcommand\smallspace[1][0.23cm]{\hspace*{#1}}
\newcommand\negativespace[1][-0.12cm]{\hspace*{#1}}

%\usepackage[T1]{fontenc} 
%\usepackage{libertine}


\begin{document}

\centerline{{\LARGE \bf Peter Sperl, B.Sc.}}
\centerline{\href{mailto:peter.sperl@outlook.com}{peter.sperl@outlook.com}}


\noindent %Prevents the following text from being indented
\\\\
\vspace*{-6pt}
{\negativespace \Large \bf Recent Professional Experience}\\
\line(1,0){485}
\\
\noindent

\noindent
{\bf Senior Engineer \& Team Lead}, \textit{employee at \href{https://anyline.com}{Anyline}} \hfill \textit{2022 -- Present}
\begin{itemize}
\setlength\itemsep{0.05em}
\item Led cross-functional development team responsible for browser-based (C++ and WebAssembly) & server-based
(NestJS) scanning solutions
\item Optimized CI/CD pipeline for faster deployments and automated testing
\item Led development on C++ Core Engine and WebAssembly SDK
\end{itemize}

\noindent
{\bf Software Architect}, \hfill \textit{2020 -- 2022} \\
\begin{itemize}
\setlength\itemsep{0.05em}
\item Reshaped the C++ engine and mobile SDKs, reducing development time for new features and standardizing scanning capabilities across all platforms.
\item Led efforts to optimize the integration of cross-platform functionalities, minimizing redundancies and streamlining feature rollouts across Android, iOS, Windows, and web platforms.
\end{itemize}

\noindent
{\bf C++ \& C\# Engineer}, \hfill \textit{2018 -- 2020} \\
\begin{itemize}
\setlength\itemsep{0.05em}
\item Ported the existing C++ engine to the Windows platform, creating a Universal Windows Platform (UWP) SDK in C\# from scratch to expand product reach.
\item Integrated our UWP SDK into a Hololens Application via Unity to create an innovative scanning capabilities
\end{itemize}

\noindent
{\bf Xamarin SDK Engineer}, \hfill \textit{2015 -- 2017} \\
\begin{itemize}
\setlength\itemsep{0.05em}
\item Developed the Anyline Xamarin SDK for Android and iOS from the ground up, establishing a solid foundation for future SDK developments.
\item Focused on mobile-based computer vision solutions, collaborating on early product iterations and building core functionality.
\end{itemize}

\noindent %Prevents the following text from being indented
\\
\\
\vspace*{-6pt}
{\negativespace \Large \bf Projects \& Developer Community Engagement}\\
\line(1,0){485}
\\
\noindent

\noindent
\textbf{Game Development:} Created multiple indie games using Unity, Monogame, Game Maker Studio and other frameworks \\
\textbf{Hackathon Participant}: Regular participant in game jams and coding competitions (15+ events) \\
\textbf{Notable Achievements}: \\
\begin{itemize}
\item \href{https://ldjam.com/events/ludum-dare/55/tao}{Tao} - 8th overall place out of 2000+ submissions (written in LUA \& pico8)
\item \href{https://www.newgrounds.com/portal/view/599044}{Llama in your Face} - A browser game reaching over 6 million players worldwide
\item \href{https://ldjam.com/users/leorean/games}{More Game Submissions}
\item \href{https://github.com/leorean/GMStudio}{Vast collection} of 50+ Game Maker-based prototypes
\item \href{https://github.com/leorean/SPG}{SPG} - A XNA-based SDK for game development
\end{itemize}

\noindent %Prevents the following text from being indented
\\
\vspace*{-6pt}
{\negativespace \Large \bf Education}\\
\line(1,0){485}\\
\\
\noindent
{\bf Bachelor of Science - Medical Informatics} \hfill \textit{2013} \\
\textit{Technische Universität Wien} \\
\\
\\

\noindent
{\bf Technologies \& Frameworks:} \textbf{UWP, .NET, Unity, Xamarin, XNA, Monogame, Game Maker Studio} \\
{\bf Languages:} \textbf{C\#, C++}, Python, Java, LUA, GML (Game Maker Language)


\end{document}
